
\documentclass{article}
\usepackage{filecontents}

% Here's the package file
\begin{filecontents*}{tagstoc.sty}
\usepackage{tocloft,etoolbox}

% Declare the master TOC.  This will contain:
% * A list of tags
% * A sequence of entries where tags are referenced
% and it will be used multiple times to generate lists of tag usage
\newlistof{game}{snip}{All Games}
\newlistof{tags}{tags}{}
\renewcommand*\cfttagstitlefont{\Large\bfseries}% For example

% Use this in the preamble to make a new tag.
\newcommand*\declaretag[1]{%
 \newlistentry[tags]{tag#1}{tags}{0}%
 \listadd\Tags{#1}%
 %\addtocontents{tags}{(#1) }%
}
% This is only true when printing the list of tags
\newif\ifprintingtags
% This makes sure that the list of tags is not printed most of the time...
\addtocontents{tags}{\protect\ifprintingtags}
% ...because it wraps the entire TOC from the preamble
\AtBeginDocument{\addtocontents{tags}{\protect\fi}}

% This is a rather inefficient way to selectively print particular tags.
% Presumably, I should just pop each tag from the list as I go,
% but etoolbox doesn't seem to handle stacks.  Probably I'm missing something.
\newcommand*\deactivatetag[1]{%
 \expandafter\let\csname l@tag#1\endcsname=\@gobbletwo
}
\newcommand*\activatetag[1]{%
 \forlistloop\deactivatetag\Tags
 \expandafter\let\csname l@tag#1\endcsname=\l@section
}

% These are hooks for the user
\providecommand*\currenttag{}
\providecommand*\listoftagstitle{}
% This prints all the references to a particular tag.
% Effectively, it's a partial ToC for that tag.
\newcommand*\dotag[1]{%
 \renewcommand*\currenttag{#1}%
 \begingroup
  \activatetag{#1}%
  \section*{\listoftagstitle}%
  \@input{\jobname.tags}%
 \endgroup
}%
% This prints the games, placed (with repeats) beneath their tags.
% It also prints a list of defined tags, more or less as an excuse
% to have \listoftags handle the .tags auxiliary ToC file.
\newcommand*\gamesbytag{%
 \forlistloop\dotag\Tags
 \activatetag{}
 \printingtagstrue
 \listoftags
}

% This is how to proclaim a game, which is what gets tagged.
\providecommand*\gamename{}
\newcommand*\gametitle{\thegame. \gamename}
\newcommand*\game[1]{%
 \refstepcounter{game}%
 \renewcommand*\gamename{#1}%
 \section*{\gametitle}%
 \addcontentsline{snip}{game}{\gametitle}
}

% In case of no hyperref
\providecommand\phantomsection{}
% This is how you place a tag beneath a game.
\newcommand*\placetag[1]{%
 \phantomsection% In case of hyperref
 \addcontentsline{tags}{tag#1}{\gametitle}%
}
\end{filecontents*}

\usepackage{tagstoc}

\renewcommand*\listoftagstitle{\currenttag}
% Preferred way to declare tags.  They will be printed in this order.

{{'\\forcsvlist\declaretag{' + ','.join(cats) + '}'}}

\title{Board Game Collection}
\author{Taylor Smith}

\begin{document}
\maketitle
\newpage
\listofgame
\newpage
 \gamesbytag
 \newpage
        
     
        {{'\game{'+game.name.replace('&','\&')+'}'}}
            
                {{'\placetag{Category: '+cat+'}'}}
            
            
            \placetag{Players: 1}
            
            
            \placetag{Players: 2}
            
            
            \placetag{Players: 3}
            
            
            \placetag{Players: 4}
            
            
            \placetag{Players: 5}
            
            
            \placetag{Players: 6}
            
            
            \placetag{Players: 7}
            
            
            \placetag{Players: 8}
            
 {{game.description.replace('&','\&').replace('#','\#').replace('$','\$').replace('{','\{').replace('}','\}')}} 
		\noindent\textbf{Categories: {{','.join(game.categories)}} } \\
		\textbf{Time to play: {{game.min_playing_time~'-'~game.max_playing_time}} } \\
		\textbf{Players: {{game.min_players~'-'~game.max_players}} } \\
		\textbf{Rating: {{game.rating_average}} }
       
\end{document}